%!TEX program = xelatex
%% start of file `template-zh.tex'.
%% Copyright 2006-2013 Xavier Danaux (xdanaux@gmail.com).
%
% This work may be distributed and/or modified under the
% conditions of the LaTeX Project Public License version 1.3c,
% available at http://www.latex-project.org/lppl/.


\documentclass[11pt,a4paper,roman]{moderncv}   % possible options include font size ('10pt', '11pt' and '12pt'), paper size ('a4paper', 'letterpaper', 'a5paper', 'legalpaper', 'executivepaper' and 'landscape') and font family ('sans' and 'roman')

% moderncv 主题
\moderncvstyle{oldstyle}                        % 选项参数是 ‘casual’, ‘classic’, ‘oldstyle’ 和 ’banking’
\moderncvcolor{purple}                          % 选项参数是 ‘blue’ (默认)、‘orange’、‘green’、‘red’、‘purple’ 和 ‘grey’
%\nopagenumbers{}                             % 消除注释以取消自动页码生成功能

% 字符编码
% \usepackage[utf8]{inputenc}                   % 替换你正在使用的编码
% \usepackage{CJKutf8}


% 调整页面出血
\usepackage[scale=0.83]{geometry}
%\setlength{\hintscolumnwidth}{3cm}           % 如果你希望改变日期栏的宽度

\usepackage{fontspec}
\usepackage{xunicode}
\usepackage{xeCJK}
\usepackage{xcolor}
\usepackage{datetime} %日期
\renewcommand{\today}{\number\year 年 \number\month 月 \number\day 日}
\setmainfont{Times New Roman}
\setsansfont{Helvetica Neue}
\setmonofont{Courier New}
\setCJKmainfont{Hiragino Sans GB}
\setCJKsansfont{STKaiti}
\setCJKmonofont{STHeiti}
%\setCJKmathfont{}
\newcommand{\linkColor}[1]{\textcolor{cyan}{#1}}

% 个人信息
\name{马申彦}{}
\title{前端工程师}                     % 可选项、如不需要可删除本行
\address{北京市·朝阳区}                          % 可选项、如不需要可删除本行
\phone[mobile]{+86~18771058712}              % 可选项、如不需要可删除本行
% \phone[fixed]{+2~(345)~678~901}               % 可选项、如不需要可删除本行
% \phone[fax]{+3~(456)~789~012}                 % 可选项、如不需要可删除本行
\email{msyfls123@gmail.com}                    % 可选项、如不需要可删除本行
\homepage{ebichu.cc/blog/about}                  % 可选项、如不需要可删除本行
% \social[GitHub]{github.com/msyfls123} % optional
% \social[Douban]{https://douc.cc/4gK66z} % optional
\extrainfo{
  \begin{itemize}
  \item \href{https://github.com/msyfls123}{\underline{GitHub 资料}}
  \item \href{https://douc.cc/4gK66z}{\underline{豆瓣主页}}
  \end{itemize}
}                 % 可选项、如不需要可删除本行
\photo[64pt][0.4pt]{msy.jpg}                  % ‘64pt’是图片必须压缩至的高度、‘0.4pt‘是图片边框的宽度 (如不需要可调节至0pt)、’picture‘ 是图片文件的名字;可选项、如不需要可删除本行
% \quote{
%   \small {
%   想了解更多? - 
%   \url{http://msyfls123.github.io}
%   \\ 部分作品展示 - 
%   \url{http://www.douban.com/photos/album/155070126/}
%   }
% }                          % 可选项、如不需要可删除本行

% 显示索引号;仅用于在简历中使用了引言
%\makeatletter
%\renewcommand*{\bibliographyitemlabel}{\@biblabel{\arabic{enumiv}}}
%\makeatother

% 分类索引
%\usepackage{multibib}
%\newcites{book,misc}{{Books},{Others}}
%----------------------------------------------------------------------------------
%            内容
%----------------------------------------------------------------------------------
\begin{document}
% \begin{CJK}{UTF8}{gbsn}                       % 详情参阅CJK文件包
\maketitle

\section{教育}
\cventry{2014.10\ –\ 2017.6}{硕士}{中国地质大学}{武汉}{\textit{4.13}}{设计学}  % 第3到第6编码可留白
\cventry{2010.9\ –\ 2014.6}{学士}{中国地质大学}{武汉}{\textit{3.62}}{工业设计}

\section{工作经历}
\cventry
{2017.7\ –\ 至今}
{前端工程师}
{\href{https://read.douban.com}{\underline{豆瓣阅读}}}
{北京}
{}
{
  \begin{itemize}
  \item 负责豆瓣阅读 web / mobile / app 内网页前端的开发测试构建上线工作
  \item 使用 \href{https://graphene-python.org/}{Graphene} 从零搭建 GraphQL 服务端,统一了书籍、用户信息的查询
  \item 在项目里推动了 TypeScript 的应用(目前 TS 占全部 JS 文件比例 1/4)
  \item 不断优化前端文件 Gulp / Webpack 构建流程,从 Python 基本迁移至 Node.js
  \item 探索适宜的异步并发解决方案(Promise, React Hooks, RxJS Observable)
  \item 2019 年开始领导前端团队,工作中执行严格的 Code Review 规范
  \end{itemize}
}
\cventry
{2016.3\ –\ 2016.7}
{前端实习生}
{\href{http://woniu.com}{\underline{蜗牛游戏}}}
{苏州}
{}
{
  \begin{itemize}
  \item 编写响应式富交互的游戏活动专题页面
  \item 参与游戏直播聚合站点及 VR 视频站点的开发
  \end{itemize}
}

% \cventry{2015.7-10}{设计实习生}{佛山尚致设计有限公司}{佛山}{\url{www.bob-id.com.cn}}{
% 参与多项平面设计及交互设计工作,利用网络推动公司形象宣传,完成了公司首个众筹产品的发布
% }
% \cventry{2014.7-8}{前端开发}{武汉七彩马科技有限公司}{武汉}{\url{7caima.com/static/service}}{
% \begin{itemize}%
% \item 参与静态展示页面的设计与制作
% \item 独立完成悬浮导航,树状流动,页面锚点滚动及旋转标语展示等JS特效
% \end{itemize}
% }
% \cventry{2013.8}{生产部实习生}{无锡协友机械制造有限公司}{无锡}{}{
% 学习机械加工的流程及工艺特点,合作完成某设备塑胶卡件的加工
% }
% \cventry{2013.7}{交流学生}{韩国建国大学夏令营}{韩国忠州}{}{
% 赴韩国参观学习两周,了解当地风土人情及先进设计技术
% }
% \cventry{2013.5-6}{获奖}{北京设计周}{北京}{}{
% 获得2013年诺基亚·绿色设计大赛银奖,作品“环保垃圾桶”在设计周上展出
% }

\section{社区}
\cvitemwithcomment
{JSConf China 2019 Speech}{响应式编程使你看得更远}{
  \href{https://2019.jsconfchina.com/}{
    \underline{网址}
  }
}
\cvitemwithcomment{InfoQ 采访}{掌控前端数据流,响应式编程让你看得更远}{
  \href{https://www.infoq.cn/article/kzyb9IEj6iyHegBNrLgd}{
    \underline{网址}
  }
}

\section{个人项目}
\cvitemwithcomment{}{基于 markdown-it 和 pdfmake 的富文本书写工具}{
  \href{https://github.com/msyfls123/MaTeX}{
    \underline{仓库}
  }  \href{https://msyfls123.github.io/MaTeX/}{
    \underline{Demo}
  }
}
\cvitemwithcomment{}{企业网站开发与配置}{
  \href{https://derui-tech.com/}{
    \underline{网址}
  }
}
% \cvitemwithcomment{2015}{中国地质大学(武汉)机电学院网站}{网页交互及前端开发}
% \cvitemwithcomment{2014}{VDP.ICIDO软件在工业设计教学中的应用}{虚拟仿真软件应用}
% \cvitemwithcomment{2013}{某免费停车概念APP原型}{原型设计制作}
% \cvitemwithcomment{2013-2014}{中地大资源学院油藏物理虚拟实验室}{三维造型及贴图设计}
% \cvitemwithcomment{2013-2014}{宜巴高速公路安全施工标准化图册}{三维造型及排版}

% \section{获奖证书}
% \cvitemwithcomment{CET6}{大学英语六级考试}{505}
% \cvitemwithcomment{NCRE}{全国计算机等级考试}{二级(C语言)}
% \cvitemwithcomment{比赛}{2013年诺基亚·绿色设计大赛 暨 龙腾奖青年设计之星}{银奖}
% \cvitemwithcomment{比赛}{2014年CUIDC·全国大学生工业设计大赛}{入围奖}
% \cvitemwithcomment{学校}{奖学金}{2012校长奖学金/2014杰美特奖学金}
% \cvitemwithcomment{学校}{荣誉称号}{2012校优秀学生干部/2013校优秀共青团员}

\section{技能}
\cvline{框架/库}{React / Backbone / Formik / RxJS / Cycle.js / Svelte}
\cvline{工具}{Gulp / Webpack / Jest(a bit)}
\cvline{语言}{TypeScript / Python / GraphQL / Rust(green hand) /
  \href{https://github.com/msyfls123/portfolio}{\underline{\LaTeX{}}}
  /
  \href{https://msyfls123.github.io/blog/images/asfaloth.jpeg}{\underline{小程序}
  }
}
\cvline{学习}{WASM /  Docker / Processing / Qt}

% \cvitem{前端开发}{\small{熟练应用HTML/CSS/Javascript编写页面,掌握jQuery/AngularJS等前端框架}}
% \cvitem{后端开发}{\small{熟悉Django/Python框架,能够部署Nginx/MySQL服务器}}
% \cvitem{数据可视化}{\small{熟悉D3.js/HightCharts/Processing等语言框架,了解H5的Canvas/SVG语法,会用\LaTeX{}进行排版}}
% \cvitem{平面设计}{\small{熟练掌握AI/Ps/Id等常用平面软件操作,能独立进行VI画册等设计}}
% \cvitem{交互设计}{\small{能使用Axure/AE/PR等软件完成APP网页等交互效果,会制作简单小视频}}
% \cvitem{三维设计}{\small{掌握Rhino/3dMax/SketchUp/CATIA等三维软件,有过虚拟仪器开发经验}}

\renewcommand{\listitemsymbol}{-}             % 改变列表符号

\section{介绍}
\cvitem{爱好}{\small{
\href{https://movie.douban.com/people/kimimama/collect}{
  \underline{电影}
}、
\href{https://www.douban.com/photos/album/1873642638}{
  \underline{做饭}
}、阅读\ 、
\href{https://www.douban.com/photos/album/155070126/}{
  \underline{设计 or 画画}
}、抖音重度使用但不成瘾者
}}
\cvitem{一句话}{\small{
\href{https://www.zhihu.com/question/30068727/answer/46664009}{
  \underline{现在站在你面前的我,并不是幻影呀}
}
}}

\vfill
{\footnotesize 更新于\today}

% 来自BibTeX文件但不使用multibib包的出版物
%\renewcommand*{\bibliographyitemlabel}{\@biblabel{\arabic{enumiv}}}% BibTeX的数字标签
% \nocite{*}
% \bibliographystyle{plain}
% \bibliography{publications}                    % 'publications' 是BibTeX文件的文件名

% 来自BibTeX文件并使用multibib包的出版物
%\section{出版物}
%\nocitebook{book1,book2}
%\bibliographystylebook{plain}
%\bibliographybook{publications}               % 'publications' 是BibTeX文件的文件名
%\nocitemisc{misc1,misc2,misc3}
%\bibliographystylemisc{plain}
%\bibliographymisc{publications}               % 'publications' 是BibTeX文件的文件名

% \clearpage\end{CJK}
\end{document}


%% 文件结尾 `template-zh.tex'.
